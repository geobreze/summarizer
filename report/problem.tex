
\chapter{ПРОБЛЕМА АВТОМАТИЗАЦИИ АНАЛИЗА НЕСТРУКТУРИРОВАННОГО ТЕКСТА}

\section{Актуальные проблемы анализа неструктурированных текстов}

В настоящее время множество систем требуются в анализе неструктурированного текста, так как почти все современные компании заинтересованы в изучении своих пользователей. Например, известная российская компания Яндекс владеет сервисом Яндекс.Маркет, где пользователи могут посмотреть цены на интересующие их товары, также посмотреть отзывы на товары и магазины, продающие эти товары. Также любой зарегистрированный пользователь может оставить отзыв на товар или магазин. Компания заинтересована в том, чтобы все отзывы были честными и были написаны настоящими людьми. Эту проблему можно сформулировать как задачу кластеризации (разбиение на группы) текста. В этом случае необходимо рассмотреть две группы отзывов: \quotes{правильные}, написанные реальными людьми, и \quotes{неправильные}, сгенерированные с помощью специальных программ или написанные рекламным агенством.

Похожая задача кластеризации текста может формулироваться следующим образом: определить какое электронное сообщение является спамом, какое сообщение содержит какую-то информацию, не являющуюся спамом и какое сообщение содержит в себе очень важную информацию. Можно предположить, что сообщеним из определенной категории соответствуют определенные ключевые слова. Например предположим, сообщение, содержащее слова \quotes{выиграть}, \quotes{миллион}, \quotes{долларов} можно однозначно отнести к категории \quotes{спам}. Однако это не так, ведь такие же ключевые слова может содержать в себе сообщение первой важности, если речь идет, например, о выигранном тендере на миллион долларов. Такие задачи требуют более глубокого анализа алгоритмов.

Не у всех компаний существуют сервисы для отзывов, в которые пользователь может написать впечатление о компании. Или такие сервисы существуют, но пользователи не делятся своим впечатлениями напрямую с компанией. Но пользователи оставляют свое мнение о товаре или услуге в социальных сетях. Обработка записей и определение настроения записи называется задачей определения тональности текста или сентимент-анализ.

К задачам интеллектуальной обработки текстов также относится построение вопросно-ответных систем, которые пытаются тексты, потенциально содержащие ответ на заданный вопрос. В задачи такой системы также входит не только поиск подходящего текста, но и извлечение части текста, отвечающей на вопрос.

Также существуют вспомогательные задачи обработки неструктурированного текста. Вот некоторые из них:
\begin{itemize}
    \item выделение предложений и абзацев;
    \item выделение слов;
    \item определение частей речи;
    \item нахождение взаимосвязи между словами;
    \item распознавание имен собственных;
    \item приведение слов к нормальной форме.
\end{itemize}

Одной из актуальных на сегодняшний день задач обработки текстов является задача подведения итогов текста. На вход поступает неструктурированный текст неопределенного объема и требуется подвести итог и кратко сказать о чем этот текст. Существует несколько подходов к решению этой задачи. Один из них предлагает выделить главные предложения в тексте, другой генерирует новые предложения, не встречающиеся в анализируемом тексте.

Также актуальной задачей обработки неструктурированных текстов является задача выделения ключевых слов, характеризующих этот текст. После выделения ключевых слов становится легче классифицировать тексты по темам, что упрощает обработку и кластеризацию неструктурированных текстов.

\section{Проблема формализации естественных языков}

Формализация языка --- задача которая появилась тогда, когда и появилась компьютерная лингвистика. Было создано несколько моделей, которые позволяли формализовать язык для определенных задач анализа, однако общей модели, которая позволила бы записать предложения на любом естественном языке (или хотя бы одном) с помощью математических формул или отношений не создали. Проблема заключается в том, что сам язык невозможно отобразить на математические понятия без определения этих понятий на естественном языке. Из такой цикличности получается, что задача формализации естественных языков является неразрешимой.

Для задач кластеризации исследователями из Google в 2013 году была разработана модель \textit{Word2Vec}, которая позволяет получать векторное представление слов на естественном языке. Представление слова в виде вектора позволяет использовать понятие расстояния векторов для определения похожести слов.

\section{Постановка задачи}

Дано: неструктурированный текст на английском языке \textit{textIn}.

Требуется разработать систему, принимающую на входе текст \textit{textIn}, которая подводит итог текста и в качестве выходных данных предоставляет краткое содержание текста \textit{summary} с помощью алгоритма TextRank. Также необходимо найти список ключевых слов \textit{keywords} с помощью эвристических алгоритмов и алгоритмов машинного обучения.

\section{Выводы}

В этой главе были получены следующе результаты:
\begin{itemize}
    \item выявлена проблема автоматизации процессов анализа неструктурированных текстов, включая подведение итогов и выделения ключевых слов;
    \item сформулирована постановка задачи на разработку программной системы, автоматизирующей процесс анализа неструктурированных документов.
\end{itemize}
\newpage
