\chapter*{ВВЕДЕНИЕ}
\addcontentsline{toc}{chapter}{ВВЕДЕНИЕ}

В глобализованном мире эффективное управление бизнес-процессами и принятие решений невозможно без анализа множества оцифрованных документов, размещенных в сотнях источников Internet (далее - сеть). Документы могут содержать жизненно важную информацию, включая мнение потребителей о качестве выпускаемой компанией продукции или услуг. Такого рода тексты формируются в различных источниках, включая цифровые издания и социальные сети. Руководство компании анализируют эти тексты и принимает решение об изменении количества или номенклатуры продукции \hyperref[itm:vissia]{[\ref{itm:vissia}]}.
К сожалению, не менее $80\%$ текстов в сети web не структурировано, т.е. не определены ключевые слова, отсутствует аннотация, не выделены логически обособленные части \hyperref[itm:vissia]{[\ref{itm:vissia}]}. В результате найти полезные для принятия решений документы в бесконечном входном потоке крайне сложно и дорого.

Один из способов сокращения времени и затрат на анализ документов заключается в автоматизации процесса структуризации документов, что сделает более простым анализ контента и понимание смысла документа. В результате разработки и применения такого рода компьютерных систем (Natural Language Processing -- NLP) появляется возможность фильтрации и разделения входного потока на полезные и ненужные документы. Анализ группы полезных документов позволяет в определенной мере выявить мнения потребителей товаров и услуг для улучшения их качества. Одна из NLP-систем -- программный комплекс BuzzTalk -- разработан компанией Byelex (Netherland) и достаточно эффективно применяется для поддержки принятия решений в крупных компаниях ЕС, США, Канады, Китая \hyperref[itm:vissia]{[\ref{itm:vissia}]}.

Крупным недостатком BuzzTalk и аналогов является высокая стоимость, сложность и большие затраты времени на разработку. Поэтому они доступны в основном крупным компаниям. Другим недостатком является их избыточность и  использование не всегда корректных результатов обработки неструктурированных текстов сторонними компаниями \hyperref[itm:vissia]{[\ref{itm:vissia}]}.  

Поэтому средним и малым компаниям, работающим, как правило, в узких секторах рынка, не выгодно или сложно использовать системы типа BuzzTalk. Соответственно, множество полезных для принятия эффективных решений документов проходят мимо руководства компаний, что снижает качество решений.
В данной работе представлен вариант системы, в определенной мере решающей эту проблему на основе синтеза компьютерных технологий и компьютерной лингвистики \hyperref[itm:ermakov]{[\ref{itm:ermakov}]}. Этому направлению посвящены труды многих известных ученых, включая А.Тьюринга, Ж.Вейзенбаума,  Д.Хадчинса, В.Ю.Розенцвейга, Г.В. Чернова, Н.Д.Андреева и др. Их работы опирались на результаты великих лингвистов А.Гумбольта, П. дю Понсо, Д.Пауэлла и др.    
В настоящее время термин компьютерная лингвистика (математическая лингвистика, вычислительная лингвистика) понимается как \quotes{направление в лингвистике, связанное с получением новой информации о тексте, речи и языке в целом с использованием компьютеров, математических методов и методов информационного моделирования} \hyperref[itm:ermakov]{[\ref{itm:ermakov}]}.

Во времена зарождения науки компьютерной лингвистики считалось, что естественные языки (те языки, которые использует человек в своей речи) можно полностью формализовать и научить компьютер их \quotes{понимать}. Однако решение задачи формализации языка оказалось слишком сложным потому что большая часть языка строится на интуиции, которой компьютер не обладает. Отсюда появилась необходимость научить компьютер работать с текстами на естественном языке без строгой формализации.

Для реализации этого подхода могут использоваться детерминированные алгоритмы, модель которых не обладает каким-либо опытом, однако с появлением Интернета и с увеличением количества доступной информации, для обработки текстов всё больше и больше используются алгоритмы, нуждающиеся в предварительном обучении. В данной работе будут рассмотрены как детерминированные алгоритмы, так и алгоритмы, требующие предварительного обучения.

В целом, курсовая работа состоит из введения, четырех глав, заключения, списка использованных источников и двух приложений.

Введение содержит описание современного состояния проблемы структуризации текста.

В первой главе выполнен анализ проблематики неструктурированных текстов и сформулирована задача курсовой работы. 

Во второй главе разработаны необходимые для решения модели и алгоритмы.

В третьей главе описана архитектура системы для структуризации текста и разработан соответствующий программный инструментарий.

В четвертой главе представлены результаты проведения экспериментов для проверки качества работы программного обеспечения.

В заключении перечислены результаты, полученные в результате выполнения курсовой работы.

Приложения содержат разработанный программный код и тексты, которые использовались для проведения экспериментов.

\newpage
