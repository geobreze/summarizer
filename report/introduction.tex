\chapter*{ВВЕДЕНИЕ}
\addcontentsline{toc}{chapter}{ВВЕДЕНИЕ}

Создание ЭВМ в середине прошлого века привело к зарождению новых наук с использованием вычислительных мощностей, предоставляемых этими машинами. В эти времена также начала зарождаться наука, которая сейчас называется компьютерной лингвистикой. Во времена зарождения этой науки считалось, что естественные языки (те языки, которые использует человек в своей речи) можно полностью формализовать и научить компьютер \quotes{понимать} человеческие языки. Однако задача формализации языка оказалась слишком сложной для её выполнения, потому что большая часть языка строится на интуиции, которой компьютер не обладает. Отсюда появилась необходимость научить компьютер работать с текстами на естественном языке без строгой формализации.

Для таких задач могут использоваться детерминированные алгоритмы, модель которых не обладает каким-либо опытом, однако с появлением Интернета и с увеличением количества доступной информации, для обработки текстов всё больше и больше используются алгоритмы, нуждающиеся в предварительном обучении. В данной работе будут рассмотрены как детерминированные алгоритмы, так и алгоритмы, требующие предварительного обучения.
\newpage
