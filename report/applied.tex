\chapter{РЕШЕНИЕ ПРИКЛАДНОЙ ЗАДАЧИ}

\section{Постановка прикладной задачи}

Разработать систему хранения  документов с выделением итога документа и выделением ключевых слов для упрощения индексации и поиска документов. 

\section{Решение прикладной задачи}

На основе разработанного программного обеспечения описанная выше прикладная задача имеет следующее решение. Научному сотруднику предлагается графический интерфейс для загрузки документов в базу данных и графический интерфейс для ознакомления со всей базой данных документов. На основе уже разработанного программного обеспечения для поиска в базе данных. Одним из таких программных продуктов является бесплатный клиент базы данных $mongoDB$ с открытым исходным кодом \textit{MongoDB Compass}\hyperref[itm:compass]{[\ref{itm:compass}]}.

Приложение, разработанное в этой главе вместе с инструментом для фильтрации результатов в базе данных позволяют организовать хранилище документов, которое позволит быстро найти нужные документы.

\section{Анализ результатов}

Пользователями системы будут являться, например, научные работники, которым необходимо обращаться к собственной, закрытой базе данных документов. Ключевые слова помогут найти статью по нужной теме, а итог текста поможет понять смысл текста. Вообще говоря, разработанное программное обеспечение может использоваться в любых сферах, где требуется хранение и поиск документов в собственной базе данных. Система является самообучаемой и при добавлении новых документов в базу данных, она корректирует коэффициенты, используемые для подсчета важности слов в тексте.

Однако система не подойдет для хранения статей или книг большого объема, так как невозможно выделить итог книги в несколько предложений. Плюсом данной системы является горизонтальная масштабируемость, которая позволяет увеличивать производительность системы с помощью добавления новых вычислительных устройств, а не улучшения старых ЭВМ.

В текущей версии программы выделение ключевых используются два алгоритма, так как эвристический алгоритм показывает лучшие результаты на текстах малого объема, а алгоритм на основе машинного обучения лучше находит ключевые слова в текстах среднего объема. В итоговом продукте можно выбирать алгоритм на основе размера входного текста, однако в текущей версии программы каждому тексту в демонстративных целях присваивается два набора ключевых слов. 

\section{Выводы}

В четвертой главе были получены следующие результаты:
\begin{itemize}
\item сформулированы прикладная задача анализа неструктурированного текста;
\item представлено ее решение на основе разработанного программного инструментария;
\item представлен анализ результатов и пути использования разработанных программ в различных прикладных областях. 
\end{itemize}
